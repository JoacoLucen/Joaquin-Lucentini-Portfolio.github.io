\documentclass[11pt, a4paper]{article}

% --- PAQUETES Y CONFIGURACIÓN ---
\usepackage[utf8]{inputenc}
\usepackage[T1]{fontenc}
\usepackage[spanish]{babel}
\usepackage[margin=0.6in]{geometry} % Márgenes un poco más estrechos para aprovechar espacio
\usepackage{xcolor}
\usepackage{parskip}   % Elimina sangría y añade espacio entre párrafos
\usepackage{fontawesome} % Iconos
\usepackage{enumitem}  % Para personalizar listas
\usepackage[hidelinks]{hyperref} % Links activos pero SIN cajas de colores

% --- FUENTE MODERNA (SANS-SERIF) ---
% Esto le da el toque "Tech/Moderno" en lugar de "Tesis Académica"
\renewcommand{\familydefault}{\sfdefault}
\usepackage{helvet}

% --- COLORES ---
\definecolor{primary}{RGB}{0, 70, 140}    % Azul corporativo serio
\definecolor{secondary}{RGB}{80, 80, 80}  % Gris oscuro para detalles
\definecolor{accent}{RGB}{100, 100, 100}  % Gris medio para fechas/lugares

% --- ESTILO DE SECCIONES ---
\usepackage{titlesec}
% Títulos en mayúsculas, color primario y línea debajo
\titleformat{\section}{\large\bfseries\color{primary}\uppercase}{}{0em}{}[\titlerule]
\titlespacing*{\section}{0pt}{14pt}{8pt} % Espaciado antes y después del título

% --- COMANDOS PERSONALIZADOS ---
% Formato para items de lista con fecha a la derecha
\newcommand{\projectHeader}[3]{
    \textbf{#1} \hfill {\footnotesize \color{accent} #2 \textbar{} #3}
}
\newcommand{\schoolHeader}[2]{
    \textbf{#1} \hfill {\footnotesize \color{accent} #2}
}

% Quitar número de página
\pagestyle{empty}

% --------------------------------------------------------------------------------------
% INICIO DEL DOCUMENTO
% --------------------------------------------------------------------------------------
\begin{document}

% --- ENCABEZADO ---
\begin{center}
    {\Huge \textbf{Joaquín Lucentini}}\\[5pt]
    {\Large \color{secondary} Ciencia de Datos en Organizaciones}\\[8pt]
    
    % Información de contacto en una sola línea limpia con separadores
    {\small
        \faMapMarker{} La Plata, BA \quad \textbullet \quad
        \faEnvelope{} \href{mailto:joacolucen96@gmail.com}{joacolucen96@gmail.com} \quad \textbullet \quad
        \faLinkedin{} \href{https://www.linkedin.com/in/joaquin-lucentini-a48066277/}{in/joaquin-lucentini}
        \\[3pt] % Segunda línea
        \faGithub{} \href{https://github.com/JoacoLucen}{github.com/JoacoLucen} \quad \textbullet \quad
        \faGlobe{} \href{https://joacolucen.github.io/Joaquin-Lucentini-Portfolio.github.io/}{Portfolio Web}
    }
\end{center}

% --- PERFIL ---
\section*{Perfil}
Estudiante de \textbf{Ciencia de Datos en Organizaciones (UNLP)}. Apasionado por la intersección entre los datos y la toma de decisiones. Cuento con una sólida base matemática y experiencia académica en el ciclo completo del dato: desde la extracción y limpieza, hasta el almacenamiento (SQL/NoSQL) y visualización. Busco aplicar mis habilidades técnicas para optimizar los procesos, la toma de decisiones y generar impacto real en un equipo profesional.

% --- SKILLS (STACK TECNOLÓGICO) ---
\section*{Stack Tecnológico}
\begin{itemize}[leftmargin=*, noitemsep]
    \item \textbf{Lenguajes:} SQL, NoSQL (Redis, Neo4j, MongoDB), Python (Librerias: Pandas, NumPy, Matplotlib, Plotly, BeautifulSoup4, Requests).
    \item \textbf{Visualización:} Streamlit, ReportLab (Generación de PDF), PowerBI.
    \item \textbf{Herramientas:} Git/GitHub, Jupyter Notebooks, VS Code.
\end{itemize}

% --- PROYECTOS ---
\section*{Proyectos Destacados}

\begin{itemize}[leftmargin=*]
    
    % --- PROYECTO 1 ---
    \item \projectHeader{Monitor de Medios Argentinos: ETL y Visualización}{Python | Streamlit}{2025}
    \begin{itemize}[label=\textbullet]
        \item Desarrollo de una solución \textit{End-to-End} de \textbf{Web Scraping} para monitorear titulares de grandes medios (TN, C5N, La Nación, Clarín).
        \item \textbf{Ingeniería de Datos:} Creación de pipelines automatizados para la limpieza de texto y detección de tópicos políticos/económicos.
        \item \textbf{Producto:} Dashboard interactivo con generación automática de reportes PDF para análisis de coyuntura.
        \item \href{https://github.com/JoacoLucen/scraping_web}{\faGithub{} \textit{\underline{Ver Código Fuente}}}
    \end{itemize}
    \vspace{4pt} % Pequeño espacio entre proyectos

    % --- PROYECTO 2 ---
    \item \projectHeader{Dashboard Analítico de la EPH (INDEC)}{Python | Pandas}{2025}
    \begin{itemize}[label=\textbullet]
        \item Procesamiento y transformación de microdatos de la Encuesta Permanente de Hogares (EPH).
        \item Diseño de visualizaciones interactivas para democratizar el acceso a indicadores socioeconómicos complejos.
        \item Despliegue de la aplicación en la nube accesible para todo público.
        \item \href{https://eph-insight-app-joacolucentini.streamlit.app/}{\faGlobe{} \textit{\underline{Ver App Online}}} \textbar{} \href{https://github.com/JoacoLucen/EPH-Insight-App}{\faGithub{} \textit{\underline{Ver Código}}}
    \end{itemize}
    \vspace{4pt}

    % --- PROYECTO 3 ---
    \item \projectHeader{Sistema de Gestión de Inversiones}{Google Sheets | API}{2025}
    \begin{itemize}[label=\textbullet]
        \item Desarrollo de una herramienta automatizada para el *tracking* de activos financieros en tiempo real.
        \item Implementación lógica del método \textbf{FIFO (First-In, First-Out)} para el cálculo exacto de rentabilidad fiscal y real.
        \item Integración con Google Finance para la actualización automática de cotizaciones y métricas de riesgo.
        \item \href{https://docs.google.com/spreadsheets/d/1MOAbafLv-NISA2nb2gy0AmwRGHegZyXV7TNal1StVgE/edit?usp=sharing}{\faTable{} \textit{\underline{Ver Planilla}}}
    \end{itemize}

\end{itemize}

% --- FORMACIÓN ACADÉMICA ---
\section*{Formación Académica}
\begin{itemize}[leftmargin=*]
    \item \schoolHeader{Ciencia de Datos en Organizaciones}{UNLP | 2024 -- Presente}
\end{itemize}

\end{document}